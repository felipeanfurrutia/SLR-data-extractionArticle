
%Document generated using DScaffolding from https://www.mindmeister.com/1167676177
\documentclass{article}
\usepackage[utf8]{inputenc}

\newcommand{\todo}[1] {\iffalse #1 \fi} %Use \todo{} command for bringing ideas back to the mind map

\title{SLR data extraction}
\author{}

\begin{document}

\maketitle
      

\section{Introduction}

%Describe the practice in which the problem addressed appears
There is a growing body of literature that recognises the importance of conducting secondary studies in Software Engineering. A key aspect of conducting secondary studies in Software Engineering is that it is Are similar to quantitative and qualitative research \cite{Wohlin2013}. Apart from that, in order to know what it is done in a specific topic or area, identify research gaps, structure better saturated areas or give recommendations. is a fundamental property of conducting secondary studies in Software Engineering. different types is another important aspect of conducting secondary studies in Software Engineering. With respect to this, it has been reported that Systematic Mapping (SM) and Systematic Literature Review (SLR). Apart from that, Most of the secondary studies are performed by novice researcher (e.g. phD students) is a fundamental property of conducting secondary studies in Software Engineering \cite{Carver2013}. main differences between SLR and SM is another important aspect of conducting secondary studies in Software Engineering. With respect to this, it has been reported that different use of the type of primary study \cite{Kitchenham2015}, a SLR aims to aggregate information from the outcomes of qualitative primary studies. \cite{Kitchenham2015} and A SLR reports a synthesis \cite{Wohlin2013}. conducting secondary studies in Software Engineering encompasses different activities: Protocol definition, Data extraction (DE) and Data Synthesis. As far as Data extraction (DE) is concerned, it has been described as Extracted data feeds next activity: classifying (in SM) and data synthesis (in SLR) \cite{Garousi2017}. 
    
%Describe the practical problem addressed, its significance and its causes
Research has shown that a major problem within conducting secondary studies in Software Engineering is that Secondary studies do not report in a clear and effective manner \cite{Budgen2018} \cite{Budgen2018}. This problem is of particular concern as it is now well established that it can lead to SLR is not reliable \cite{Garousi2017} \cite{Wohlin2013}, the aggregations in codified form are hard to read and understand \cite{Dieste2008}, Little confidence about the findings, It is difficult to asses the scientific credibility of its results by SLR  readers and SLR results are difficult to compare. Causes can be diverse: (1) lack of internal/external validation \cite{Ampatzoglou2019}, (2) reviewers do not know which data is relevant \cite{Brereton2011}, (3) reviewers do not understand data extraction requirements \cite{Brereton2007} \cite{Zhou2017}, (4) innacurate or missing data: insufficient information for the extraction process \cite{Zhou2017}, (5) researcher is novice \cite{Ribeiro2018} \cite{Carver2013}, (6) relevant information is spread among different sections \cite{Budgen2018} \cite{Budgen2018}, (7) Extensive body of knowledge to be screened, classified and synthesized \cite{link.springer.com}, (8) effort to meet consensus (when many reviewers take part in data-extraction phase) \cite{Ribeiro2018} \cite{Riaz2010} \cite{Brereton2007} \cite{Staples2007} \cite{Bandara2015}, (9) too much time finding evidences, (10) too many studies to review \cite{Garousi2017}, (11) agenda problem and too many meetings \cite{Staples2007} \cite{Staples2007}, (12) geographically distributed-teams \cite{Riaz2010}, (13) researcher's bias \cite{Imtiaz2013} \cite{Ribeiro2018} \cite{Zhou2017} \cite{Zhou2017}, (14) data problems \cite{Bandara2015}, (15) data extraction forms are fulfilled manually or require too many mouse clicks \cite{Staples2007} \cite{Ramezani2017} \cite{Ramezani2017} \cite{Bandara2015} \cite{LU2008} and (16) Current specific software related to SLR and CAQDAS are not used. 
    
%Summarise existing research including knowledge gaps and give an account for similar and/or alternative solutions to the problem
Existing research has tackled these causes. Petersen et al. addressed the too much time finding evidences \cite{Petersen2015}. Garousi et al. addressed the too much time finding evidences \cite{Garousi2017}. However, this approach has the following limitation: they are fulfilled manually. Garousi et al. addressed the geographically distributed-teams \cite{Garousi2017}. LU et al. addressed the data extraction forms are fulfilled manually or require too many mouse clicks \cite{LU2008}. Lu et al. addressed the data extraction forms are fulfilled manually or require too many mouse clicks \cite{Lu2008}. However, this approach has the following limitation: data format requirements \cite{Lu2008}. 
    
%Formulate goals and present Kernel theories used as a basis for the artefact design
In this work, we address 5 main causes: peer review is effort intensive and also takes more calendar time, too much time finding evidences, geographically distributed-teams, data problems and data extraction forms are fulfilled manually or require too many mouse clicks. To lessen these causes, we resort to provides a basis for establishing credibility and validity, make the examination of qualitative data more complete and rigorous, it is a requirement from the community, higher effectiveness, faciltate information sharing and knowledge exchange and interfaces to transfer data among tools. 
    
%Describe the kind of artefact that is developed or evaluated
This article presents a novel artefact
    
%Formulate research questions

    
%Summarize the contributions and their significance

      
%Overview of the research strategies and methods used
This article has followed a Design Science Research approach.

%Describe the structure of the paper
The remainder of the paper is structured as follows: 

%Optional - illustrate the relevance and significance of the problem with an example
    
      
\bibliographystyle{unsrt}
\bibliography{references}

\end{document}
    